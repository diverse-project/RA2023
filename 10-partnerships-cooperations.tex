% [radar] -----------------------------------
% [radar] Do not alter this section title
\section{Partnerships and cooperations}
\label{diverse:partnerships}
% [radar] -----------------------------------
\subsection{International initiatives}

\subsubsection{Associate Teams in the framework of an Inria International Lab or in the framework of an Inria International Program}

\subsubsection{Inria associate team not involved in an IIL or an international program}


\paragraph{RESIST\_EA}
% Automatic import, you may need to refine the content!
%  [BEGIN DRISI (drisi.inria.fr) IMPORT] 
\begin{description}
  \item[Title:] Resilient Software Science
  \item[Duration:] 2021 -> 
  \item[Coordinator:] Arnaud Gotlieb (arnaud@simula.no)
  \item[Partners:] \leavevmode 
  \begin{itemize}
    \item SIMULA (Norvège)
  \end{itemize}
  \item[Inria contact:] \pers{Mathieu}{Acher}
\item[Summary:]The Science of Resilient Software (RESIST\_EA) intends to create software-systems which can resist failures without significantly degrading their functionality. For several years, creating resilient software-systems has become extremely important in various application domains. For example, in robotics, the deployment of advanced collaborative robots which have to cope with uncertainty and unexpected behaviors while being able to recover from their failures has led to new research challenges. A recent area where these challenges have become pregnant is industrial robotics for car manufacturing where major issues faced by an “excessive automation” have surfaced. For instance, Tesla has struggled with painting, welding, assembling industrial robots in its advanced California car factory since 2018. Generally speaking, Autonomous Software-Systems (AS) such as self-driving cars, autonomous ships or industrial robots require the development of resilient software-systems as they have to manage unexpected events, such as faults or hazards.
The goal of the Associate Team “Resilient Software Science” (and the main innovation of this project) is to explore the Science of resilient software by laying the ground to foundational work on advanced a priori testing methods such as metamorphic testing and a posteriori continuous improvements through digital twins.
\end{description}

% [END DRISI IMPORT]

\subsubsection{STIC/MATH/CLIMAT AmSud projects}

\subsubsection{Participation in other International Programs}

% [doc] For each collaboration please specify the Program: 
% [doc]  (CNRS (IEA, IRN, IRP, IRL formerly known as  LIA, UMI, PICS, PRC, GDRI/IRN) 
% [doc]  PHC (Partenriat Hubert Curien) // ECOS Nord/ Ecos Sud/ COFECUB // ANR International 
% [doc] // Other to specify) and the information below:

% \paragraph{ Projet Acronym}

%% Please add the team participants for this project:
% \begin{participants}
%    \pers{surname1}{lastname1}
%    \pers{surname2}{lastname2}
% \end{participants}


% \begin{description}
    %     \item[Title:]
    %     \item [Partner Institution(s):]
    %     \begin{itemize}
    %         \item Institution name, Country
    %         ...
    %     \end{itemize}
    %     \item[Date/Duration:]
    %     \item[Additionnal info/keywords:]
    % \end{description}
    \subsection{International research visitors}

\subsubsection{Visits of international scientists}

\paragraph{Inria International Chair}

%% Please add the team participants for this project:
% \begin{participants}
%    \pers{surname1}{lastname1}
%    \pers{surname2}{lastname2}
% \end{participants}


\paragraph{Other international visits to the team}
% [doc] Please use the following model:

% \subparagraph{Name of the researcher}
% \begin{description}
%     \item[Status] (researcher, PhD, post-Doc, intern (master/eng))
%     \item[Institution of origin:] 
%     \item[Country:] 
%     \item[Dates:] 
%     \item[Context of the visit:] 
%     \item[Mobility program/type of mobility:] (sabbatical, internship, research stay, lecture…)
% \end{description}

\subsubsection{Visits to international teams}
\paragraph{Sabbatical programme}
\paragraph{Research stays abroad}

% [doc] Please use the following model:

% \subparagraph{pers{surname}{lastname}}
% \begin{description}
%     \item[Visited institution:] 
%     \item[Country:] 
%     \item[Dates:] 
%     \item[Context of the visit:] 
%     \item[Mobility program/type of mobility:] (sabbatical, internship, research stay, lecture…)
% \end{description}

\subsection{European initiatives}
\subsubsection{Horizon Europe}

\paragraph{HiPEAC}


%% Please add the team participants for this project:
% \begin{participants}
%    \pers{surname1}{lastname1}
%    \pers{surname2}{lastname2}
% \end{participants}


% Automatic import, you may need to refine the content!
%  [BEGIN CORDIS(https://cordis.europa.eu) IMPORT] 
\href{https://dx.doi.org/10.3030/101069836}{HiPEAC project on cordis.europa.eu}
\begin{description}
  \item[Title:] High Performance, Edge And Cloud computing
  \item[Duration:] From December 1, 2022 to May 31, 2025
  \item[Partners:] \leavevmode
  \begin{itemize}
    \item INSTITUT NATIONAL DE RECHERCHE EN INFORMATIQUE ET AUTOMATIQUE (INRIA), France
    \item ECLIPSE FOUNDATION EUROPE GMBH (EFE GMBH), Germany
    \item INSIDE, Netherlands
    \item UNIVERSITEIT GENT (UGent), Belgium
    \item RHEINISCH-WESTFAELISCHE TECHNISCHE HOCHSCHULE AACHEN (RWTH AACHEN), Germany
    \item COMMISSARIAT A L ENERGIE ATOMIQUE ET AUX ENERGIES ALTERNATIVES (CEA), France
    \item SINTEF AS (SINTEF), Norway
    \item IDC ITALIA SRL, Italy
    \item THALES (THALES), France
    \item CLOUDFERRO SA, Poland
    \item BARCELONA SUPERCOMPUTING CENTER CENTRO NACIONAL DE SUPERCOMPUTACION (BSC CNS), Spain
   \end{itemize}
   \item[Inria contact:] \pers{Olivier}{Zendra}
  \item[Coordinator:] % please add the project's coordinator name
  \item[Summary:] The objective of HiPEAC is to stimulate and reinforce the development of the dynamic European computing ecosystem that supports the digital transformation of Europe. It does so by guiding the future research and innovation of key digital, enabling, and emerging technologies, sectors, and value chains.  The longer term goal is to strengthen European leadership in the global data economy and to accelerate and steer the digital and green transitions through human-centred technologies and innovations. This will be achieved via mobilising and connecting European partnerships and stakeholders to be involved in the research, innovation and development of computing and systems technologies. They will provide roadmaps supporting the creation of next-generation computing technologies, infrastructures, and service platforms.



The key aim is to support and contribute to rapid technological development, market uptake and digital autonomy for Europe in advanced digital technology (hardware and software) and applications across the whole European digital value chain. HiPEAC will do this by connecting and upscaling existing initiatives and efforts, by involving the key stakeholders, and by improving the conditions for large-scale market deployment. The next-generation computing and systems technologies and applications developed will increase European autonomy in the data economy. This is required to support future hyper-distributed applications and provide new opportunities for further disruptive digital transformation of the economy and society, new business models, economic growth, and job creation.



The HiPEAC CSA proposal directly addresses the research, innovation, and development of next generation computing and systems technologies and applications. The overall goal is to support the European value chains and value networks in computing and systems technologies across the computing continuum from cloud to edge computing to the Internet of Things (IoT).
\end{description}
% [END CORDIS IMPORT]

\subsubsection{H2020 projects}
\subsubsection{Digital Europe}
\subsubsection{Other european programs/initiatives}


%% [begin RA2022 imported content]
\subsection{National initiatives}
\label{diverse:national-initiatives}

 \subsubsection{ANR}

\paragraph*{\label{project:MC-Evo}MC-Evo2 ANR JCJC}


\begin{participants}
\pers{Djamel Eddine}{Khelladi}
\end{participants}

\begin{itemize}
	\item Coordinator: Djamel E. Khelladi
	\item DiverSE, CNRS/IRISA Rennes
	\item Dates: 2021-2025
	\item Abstract:
  Software maintenance represents 40\% to 80\% of the total cost of developing software. On 65~projects, an IT company reported a cost of several million dollars, with a 25\% higher cost on complex projects. Nowadays, software evolves frequently with the philosophy “Release early, release often” embraced by IT giants like the GAFAM, thus making software maintenance difficult and costly. Developing complex software inevitably requires developers to handle multiple dimensions, such as APIs to use, tests to write, models to reason with, etc. When software evolves, a co-evolution is usually necessary as a follow-up, to resolve the impacts caused by the evolution changes. For example, when APIs evolve, code must be co-evolved, or when code evolves, its tests must be co-evolved. The goals of this project are to: 1)~address these challenges from a novel perspective, namely a multidimensional co-evolution approach, 2)~investigate empirically the multidimensional co-evolution in practice in GitHub, Maven, and Eclipse, 3)~automate and propagate the multidimensional co-evolution between the software code, APIs, tests, and models.
\end{itemize}



\subsubsection{DGA}

\paragraph*{\label{project:fpml}LangComponent (CYBERDEFENSE)}

\begin{participants}
\pers{Benoît}{Combemale}
\pers{Olivier}{Barais}
\end{participants}

   \begin{itemize}
   	    \item Coordinator: DGA
      	\item Partners: DGA MI, INRIA
      	\item Dates: 2019-2022
      	\item Abstract: in the context of this project, DGA-MI and the INRIA team DiverSE explore the existing approaches to ease the development of formal specifications of domain-Specific Languages (DSLs) dedicated to packet filtering, while guaranteeing expressiveness, precision and safety. In the long term, this work is part of the trend to provide to DGA-MI and its partners a tooling to design and develop formal DSLs which ease the use while ensuring a high level of reasoning.
      \end{itemize}


\subsubsection{DGAC}

\paragraph*{\label{project:oneway}OneWay}
\begin{participants}
\pers{Benoît}{Combemale}
\pers{Didier}{Vojtisek}
\pers{Olivier}{Barais}
\pers{Jean-Marc}{Jézéquel}
\pers{Mathieu}{Acher}
\end{participants}

   \begin{itemize}
   	    \item Coordinator: Airbus
      	\item Partners: Airbus, Dassault Aviation, Liebherr Aerospace, Safran Electrical Power, Safran
Aerotechnics, Thales, Altran Technologies, Cap Gemini, Sopra Steria, CIMPA, IMT Mines Ales, University of
Rennes 1, ENSTA Bretagne, and PragmaDev.
      	\item Dates: 2021-2022
      	\item Abstract: The ONEWAY project aims at maturing digital functional bricks for the following capacities: 1)~Digitalization, MBSE modeling and synthetic analysis by substitution model, of all the information and under all the points of view necessary for the design and validation across an extended enterprise of the complete aircraft system and at all its levels of decomposition, 2)~Generic and instantiable configuration management throughout the life cycle, on products and their support systems, in product lines or on aircraft programs, interactively in the context of an extended enterprise, 3)~Decision support for launching, then controlling and steering a Product Development Plan interactively in the context of an extended enterprise, and 4)~Helping the efficiency of IVVQ activities: its operations, its testing and data processing resources, its ability to perform massive testing.
      \end{itemize}
      
\paragraph*{\label{project:mip4}MIP 4.0}
\begin{participants}
\pers{Benoît}{Combemale}
\pers{Didier}{Vojtisek}
\pers{Olivier}{Barais}
\end{participants}

   \begin{itemize}
   	    \item Coordinator: Safran
      	\item Partners: Safran, Akka, Inria.
      	\item Dates: 2022-2023
      	\item Abstract: The MIP 4.0 project aims at investigating integrated methods for efficient and shared propulsion systems. Inria explore new techniques for collaborative modeling over the time. 
      \end{itemize}

 %TODO (djamel) Add JCJC Djamel
 

%% [end RA2022 imported content]
\subsection{Regional initiatives}
