\subsection{New software}
\label{softwares}
%% New Software and Plateform - exported content
%%
\subsubsection{FAMILIAR}
\label{bil-1468}
\begin{description}
% No Name
\item[Keywords:] Software line product, Configators, Customisation
\item[Scientific Description:]
FAMILIAR (for FeAture Model scrIpt Language for manIpulation and Automatic Reasoning) is a language for importing, exporting, composing, decomposing, editing, configuring, computing "diffs", refactoring, reverse engineering, testing, and reasoning about (multiple) feature models. All these operations can be combined to realize complex variability management tasks. 
A comprehensive environment is proposed as well as integration facilities with the Java ecosystem.
\item[Functional Description:]
Familiar is an environment for large-scale product customisation. From a model of product features (options, parameters, etc.), Familiar can automatically generate several million variants. These variants can take many forms: software, a graphical interface, a video sequence or even a manufactured product (3D printing). Familiar is particularly well suited for developing web configurators (for ordering customised products online), for providing online comparison tools and also for engineering any family of embedded or software-based products.

%No Release Contributions


%No News of the Year

\item[URL:] \url{http://familiar-project.github.com}
% No Publication
% No Author
\item[Contact:] Mathieu Acher
\item[Participants:] Mathieu Acher, Olivier Barais, Didier Vojtisek
% No Partner
\end{description}

\subsubsection{GEMOC Studio}
\label{bil-2742}
\begin{description}
\item[Name:] GEMOC Studio
\item[Keywords:] DSL, Language workbench, Model debugging
\item[Scientific Description:]
The language workbench put together the following tools seamlessly integrated to the Eclipse Modeling Framework (EMF):

 1)  Melange, a tool-supported meta-language to modularly define executable modeling languages with execution functions and data, and to extend (EMF-based) existing modeling languages.
 2)  MoCCML, a tool-supported meta-language dedicated to the specification of a Model of Concurrency and Communication (MoCC) and its mapping to a specific abstract syntax and associated execution functions of a modeling language.
 3)  GEL, a tool-supported meta-language dedicated to the specification of the protocol between the execution functions and the MoCC to support the feedback of the data as well as the callback of other expected execution functions.
 4) BCOoL, a tool-supported meta-language dedicated to the specification of language coordination patterns to automatically coordinates the execution of, possibly heterogeneous, models.
 5) Monilog, an extension for monitoring and logging executable domain-specific models
 6) Sirius Animator, an extension to the model editor designer Sirius to create graphical animators for executable modeling languages.
\item[Functional Description:]
The GEMOC Studio is an Eclipse package that contains components supporting the GEMOC methodology for building and composing executable Domain-Specific Modeling Languages (DSMLs). It includes two workbenches:
    The GEMOC Language Workbench: intended to be used by language designers (aka domain experts), it allows to build and compose new executable DSMLs.
    The GEMOC Modeling Workbench: intended to be used by domain designers to create, execute and coordinate models conforming to executable DSMLs. The different concerns of a DSML, as defined with the tools of the language workbench, are automatically deployed into the modeling workbench. They parametrize a generic execution framework that provides various generic services such as graphical animation, debugging tools, trace and event managers, timeline.

%No Release Contributions


%No News of the Year

\item[URL:] \url{http://gemoc.org/studio.html}
\item[Publications:] \href{https://hal.inria.fr/hal-00850770}{hal-00850770}, \href{https://hal.inria.fr/hal-01355391}{hal-01355391}, \href{https://hal.inria.fr/hal-01609576}{hal-01609576}, \href{https://hal.inria.fr/hal-01651801}{hal-01651801}, \href{https://hal.inria.fr/hal-01152342}{hal-01152342}, \href{https://hal.inria.fr/hal-03374955}{hal-03374955}, \href{https://hal.inria.fr/hal-01614561}{hal-01614561}, \href{https://hal.inria.fr/hal-01616154}{hal-01616154}
% No Author
\item[Contact:] Benoît Combemale
\item[Participants:] Didier Vojtisek, Dorian Leroy, Erwan Bousse, Fabien Coulon, Julien DeAntoni
\item[Partners:] IRIT, ENSTA, I3S, OBEO, Thales TRT
\end{description}

\subsubsection{Interacto}
\label{bil-2542}
\begin{description}
% No Name
\item[Keywords:] GUI (Graphical User Interface), User Interfaces, HCI, Software engineering

%No Scientific Description

\item[Functional Description:]
Interacto is a framework for developing user interfaces and user interactions. It complements other general graphical framework by providing a fluent API specifically designed to process user interface event and develop complex user interactions. Interacto is currently developped in Java and TypeScript to target both Java desktop applications (JavaFX) and Web applications (Angular).

%No Release Contributions


%No News of the Year

\item[URL:] \url{https://interacto.github.io}
\item[Publications:] \href{https://hal.inria.fr/hal-03231669}{hal-03231669}, \href{https://hal.inria.fr/tel-02354530}{tel-02354530}, \href{https://hal.inria.fr/inria-00590891}{inria-00590891}, \href{https://hal.inria.fr/inria-00477627}{inria-00477627}
% No Author
\item[Contact:] Arnaud Blouin
\item[Participants:] Arnaud Blouin, Olivier Beaudoux
% No Partner
\end{description}

\subsubsection{ALE}
\label{bil-3337}
\begin{description}
\item[Name:] Action Language for Ecore
\item[Keywords:] Meta-modeling, Executable DSML

%No Scientific Description

\item[Functional Description:]
Main features of ALE include:

\begin{itemize}
\item Executable metamodeling: Re-open existing EClasses to insert new methods with their implementations
\item Metamodel extension: The very same mechanism can be used to extend existing Ecore metamodels and insert new features (eg. attributes) in a non-intrusive way
\item Interpreted: No need to deploy Eclipse plugins, just run the behavior on a model directly in your modeling environment
\item Extensible: If ALE doesn’t fit your needs, register Java classes as services and invoke them inside your implementations of EOperations.
\end{itemize}

%No Release Contributions


%No News of the Year

\item[URL:] \url{http://gemoc.org/ale-lang/}
% No Publication
% No Author
\item[Contact:] Benoît Combemale
% No Participant
\item[Partner:] OBEO
\end{description}

\subsubsection{Melange}
\label{bil-2731}
\begin{description}
\item[Name:] Melange
\item[Keywords:] Modeling language, Meta-modelisation, Language workbench, Dedicated langage, Model-driven software engineering, DSL, MDE, Meta model, Model-driven engineering, Meta-modeling
\item[Scientific Description:]
Melange is a follow-up of the executable metamodeling language Kermeta, which provides a tool-supported dedicated meta-language to safely assemble language modules, customize them and produce new DSMLs. Melange provides specific constructs to assemble together various abstract syntax and operational semantics artifacts into a DSML. DSMLs can then be used as first class entities to be reused, extended, restricted or adapted into other DSMLs. Melange relies on a particular model-oriented type system that provides model polymorphism and language substitutability, i.e. the possibility to manipulate a model through different interfaces and to define generic transformations that can be invoked on models written using different DSLs. Newly produced DSMLs are correct by construction, ready for production (i.e., the result can be deployed and used as-is), and reusable in a new assembly.

Melange is tightly integrated with the Eclipse Modeling Framework ecosystem and relies on the meta-language Ecore for the definition of the abstract syntax of DSLs. Executable meta-modeling is supported by weaving operational semantics defined with Xtend. Designers can thus easily design an interpreter for their DSL in a non-intrusive way. Melange is bundled as a set of Eclipse plug-ins.
\item[Functional Description:]
Melange is a language workbench which helps language engineers to mashup their various language concerns as language design choices, to manage their variability, and support their reuse. It provides a modular and reusable approach for customizing, assembling and integrating DSMLs specifications and implementations.

%No Release Contributions


%No News of the Year

\item[URL:] \url{http://melange-lang.org}
% No Publication
% No Author
\item[Contact:] Benoît Combemale
\item[Participants:] Arnaud Blouin, Benoît Combemale, David Mendez Acuna, Didier Vojtisek, Dorian Leroy, Erwan Bousse, Fabien Coulon, Jean-Marc Jezequel, Olivier Barais, Thomas Degueule
% No Partner
\end{description}

