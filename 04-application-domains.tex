

% [doc] -----------------------------
% [doc] Instructions
% [doc] -----------------------------
% [doc] https://intranet.inria.fr/Vie-scientifique/Information-edition-scientifiques/RADAR/Structure-du-rapport
% [doc] -----------------------------

%% [BEGIN last year imported content]






% [radar] -----------------------------------
% [radar] Do not alter this section title
\section{Application domains}
\label{diverse:domain}
% [radar] -----------------------------------
Information technology affects all areas of society. The need to develop software systems is therefore present in a huge  number of application domains. One of the goals of software engineering is to \textit{apply a systematic, disciplined, quantifiable approach to the development, operation, and maintenance of software} whatever the application domain.

As a result, the team covers a wide range of application domains and never refrains from exploring a particular field of application.
Our primary expertise is in complex, heterogeneous and distributed systems. While we historically collaborated with partners in the field of systems engineering, it should be noted that for several years now, we have investigated several new areas in depth:
\begin{itemize}
\item the field of web applications, with the associated design principles and architectures, for applications ranging from cloud-native applications to the design of modern web front-ends.
\item the field of scientific computing in connection with the CEA DAM, Safran and scientists from other disciplines such as the ecologists of the University of Rennes. In this field where the writing of complex software is common, we explore how we could help scientists to use software engineering approach, in particular, the use of SLE and approximate computing techniques.
\item the field of large software systems such as the Linux kernel or other open-source projects. In this field, we explore, in particular, the variability management, the support of co-evolution and the use of polyglot approaches. 

\end{itemize} 





%% [END last year imported content]
