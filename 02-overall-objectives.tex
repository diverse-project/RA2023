

% [doc] -----------------------------
% [doc] Instructions
% [doc] -----------------------------
% [doc] https://intranet.inria.fr/Vie-scientifique/Information-edition-scientifiques/RADAR/Structure-du-rapport
% [doc] -----------------------------

%% [BEGIN last year imported content]






% [radar] -----------------------------------
% [radar] Do not alter this section title
\section{Overall objectives}
\label{diverse:context}
% [radar] -----------------------------------



%% ---------------------------------------
%\subsection{Overall objectives}
%\label{presentation:overall}
%% ---------------------------------------


  \team{}'s research agenda targets core values of software engineering. 
  In this fundamental domain we focus on and develop models, methodologies and theories to address major challenges raised by the emergence of several forms of diversity in the design, deployment and evolution of software-intensive systems.
Software diversity has emerged as an essential phenomenon in all application domains borne by our industrial partners. These application domains range from complex systems brought by systems of systems (addressed in collaboration with Thales, Safran, CEA and DGA) and Instrumentation and Control (addressed with EDF) to pervasive combinations of Internet of Things and Internet of Services (addressed with TellU and Orange) and tactical information systems (addressed in collaboration with civil security services).
  Today these systems seem to be all  radically  different, but we envision a strong convergence  of the scientific principles that underpin their construction and validation, bringing forwards sane and reliable methods for the design of  \textbf{flexible and open yet dependable systems}. 
  Flexibility and openness are both critical and challenging software layer properties that must deal with the following four dimensions of diversity: \textbf{diversity of languages}, used by the stakeholders involved in the construction of these systems;  \textbf{diversity of features}, required by the different customers; \textbf{diversity of runtime environments}, where software has to run and adapted; \textbf{diversity of implementations}, which are necessary for resilience by redundancy.

  In this context, the central software engineering challenge consists in handling \textbf{diversity} from variability in requirements and design to heterogeneous and dynamic execution environments. 
  In particular, this requires considering that the software system must adapt, in unpredictable yet valid ways, to changes in the requirements as well as in its environment. 
  Conversely, explicitly handling diversity is a great opportunity to allow software to spontaneously explore alternative design solutions, and to mitigate security risks.
  
  Concretely, we want to provide software engineers with the following abilities:
  \begin{itemize}\itemsep0cm
      \item to characterize an ``envelope'' of possible variations;
      \item to compose envelopes (to discover new macro correctness envelopes in an opportunistic manner);
      \item to dynamically synthesize software inside a given envelope.
  \end{itemize}

  The major scientific objective that we must achieve to provide such mechanisms for software engineering is summarized below:

    \textbf{Scientific objective for \team{}:} To automatically \textbf{compose and synthesize software diversity} from design to runtime to \textbf{address unpredictable evolution of software-intensive systems} 

  Software product lines and associated variability modeling formalisms represent an essential aspect of software diversity, which we already explored in the past, and this aspect stands as a major foundation of \team{}'s research agenda. 
  However, \team{}  also exploits other foundations to handle new forms of diversity: type theory and models of computation for the composition of languages; distributed algorithms and pervasive computation to handle the diversity of execution platforms;  functional and qualitative randomized transformations to synthesize diversity for robust systems.






%% [END last year imported content]
