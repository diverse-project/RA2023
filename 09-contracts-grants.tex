

% [doc] -----------------------------
% [doc] Instructions
% [doc] -----------------------------
% [doc] https://intranet.inria.fr/Vie-scientifique/Information-edition-scientifiques/RADAR/Structure-du-rapport
% [doc] -----------------------------

%% [BEGIN last year imported content]






% [radar] -----------------------------------
% [radar] Do not alter this section title
\section{Bilateral contracts and grants with industry}
\label{diverse:contracts-grants}
% [radar] -----------------------------------


% Suggested subsection 
\subsection{Bilateral contracts with industry}
\label{diverse:contracts}

%% BC: TO BE CHECK THIS PART



\paragraph*{ADR Nokia}

\begin{participants}
\pers{Olivier}{Barais}
\pers{Johann}{Bourcier}
\end{participants}


\begin{itemize}\itemsep0cm
	\item Coordinator: Inria
	\item Dates: 2017-2021
	\item Abstract: The goal of this project is to integrate chaos engineering principles to IoT Services frameworks to improve the robustness of the software-defined network services using this approach;  to explore the concept of equivalence for software-defined network services; and to propose an approach to constantly evolve the attack surface of the network services.
\end{itemize}

\paragraph*{SLIMFAST}

\begin{participants}
\pers{Mathieu}{Acher}
\end{participants}


\begin{itemize}\itemsep0cm
	\item Partners: DGA
	\item Dates: 2021-2022
	\item Abstract: Debloating software variability for improving non-functional properties (e.g. security)
\end{itemize}



\paragraph*{BCOM}

\begin{participants}
\pers{Olivier}{Barais}
\end{participants}



\begin{itemize}\itemsep0cm
	\item Coordinator: UR1
	\item Dates: 2018-2024
	\item Abstract: The aim of the Falcon project is to investigate how to improve the resale of available resources in private clouds to third parties. In this context, the collaboration with DiverSE mainly aims at working on efficient techniques for the design of consumption models and resource consumption forecasting models. These models are then used as a knowledge base in a classical autonomous loop.  
	
\end{itemize}

\paragraph*{Debug4Science}
\begin{participants}
\pers{Benoît}{Combemale}
\end{participants}

\begin{itemize}\itemsep0cm
	\item Partners: Inria/CEA DAM
	\item Dates: 2020-2022
	\item Abstract: Debug4Science aims to propose a disciplined approach to develop domain-specific debugging facilities for Domain-Specific Languages within the context of scientific computing and numerical analysis. Debug4Science is a bilateral collaboration (2020-2022), between the CEA DAM/DIF and the DiverSE team at Inria.
\end{itemize}

\paragraph*{Orange}
\begin{participants}
\pers{Olivier}{Barais}
\pers{Benoît}{Combemale}
\pers{Stéphanie}{Chalita}
\end{participants}

\begin{itemize}\itemsep0cm
	\item Partners: UR1/Orange
	\item Dates: 2020-2023
	\item Abstract: Context aware adaptive authentification,  Anne Bumiller's PhD Cifre project.
\end{itemize}


\paragraph*{Obeo}
\begin{participants}
\pers{Benoît}{Combemale}
\pers{Arnaud}{Blouin}
\end{participants}

\begin{itemize}\itemsep0cm
	\item Partners: UR1/Obéo
	\item Dates: 2022-2025
	\item Abstract: Low-code language workbench, Theo Giraudet's PhD Cifre project.	 
\end{itemize}


\paragraph*{SAP}
\begin{participants}
\pers{Olivier}{Barais}
\end{participants}
\begin{itemize}\itemsep0cm
	\item Partners: UR1/SAP
	\item Dates: 2021-2024
	\item Abstract: Research focusing on Open-source software Supply Chain security. Piergiorgio Ladisa's PhD Cifre project.	 
\end{itemize}

% Suggested subsection 
%\subsection{Bilateral grants with industry}
%\label{diverse:grants}



%% [END last year imported content]
